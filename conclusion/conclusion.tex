\section{Conclusion}

In this report I described why it is desirable to detect double-unlock errors in the Linux kernel source code, while also presenting a prototype implementation of such a double-unlock checker on top of the existing EBA analyzer for checking the Linux kernel source code. In line with the existing implementation of the EBA analyzer, the double-unlock error has been expressed as a shape-and-effect analysis using Computational Tree Logic. The implementation details of the prototype as a new checker on top of the existing implementation of the EBA analyzer have furthermore been described and the prototype has been evaluated on a presented set of true positives found in the Linux kernel.

\newpar The developed prototype has been evaluated on 15 true positives and shows a 33\% detection rate. Detecting 5 out of 15 errors shows that such a checker helps detect errors in the Linux kernel albeit while leaving room for improvement. The limitations in the current implementation of the underlying EBA analyzer has been identified and described as the culprit of the 66\% false negative rate of the prototype along with ways to improve the detection rate in the future. Reformulating the existing checkers in the implementation of the EBA analyzer as monitor state machines and extending the parsing capabilities of the implementation have been pinpointed as possible steps to improve accuracy.

\newpar This project has shown that an error checker for the Linux kernel can be developed with relative ease by extending the capabilities of the implementation of the EBA shape-and-effect analysis. The prototype is able to detect confirmed double-unlock errors in the Linux kernel source code and can be extended in order to improve its accuracy.