\subsection{True Positives}

\subsubsection{\texttt{orphan.c}}

\newpar The patch \texttt{4dd75b33}\footnote{See Fig. \ref{truepositives} in Appendix.} patches a double unlock error in the file \texttt{fs/ubifs/orphan.c}, part of the file system components of the kernel.

\newpar The function \texttt{orphan\_delete} contains two \texttt{if} statements which, if evaluating to \texttt{true} will result in a double unlock of \texttt{\&c->orphan\_lock}, since the function \texttt{ubifs\_delete\_orphan} calls \texttt{orphan\_delete} and then unlock the same pointer value. 

\newpar The full code snippet for this function can be seen in Fig. \ref{fig:orphan.c}.

\begin{figure}[H]
    \centering
    \begin{minted}{c}
        void ubifs_delete_orphan(struct ubifs_info *c, ino_t inum)
        {
            [...]
            
            orphan_delete(c, orph);

            spin_unlock(&c->orphan_lock);
        }


        static void orphan_delete(struct ubifs_info *c, 
            struct ubifs_orphan *orph)
        {
            if (orph->del) {
                spin_unlock(&c->orphan_lock); /* If executed, double unlock */
                dbg_gen("deleted twice ino %lu", orph->inum);
                return;
            }
            if (orph->cmt) {
                orph->del = 1;
                orph->dnext = c->orph_dnext;
                c->orph_dnext = orph;
                spin_unlock(&c->orphan_lock); /* If executed, double unlock */
                dbg_gen("delete later ino %lu", orph->inum);
                return;
            }
        }
    \end{minted}
    \caption{The \texttt{orphan\_delete} function containing a double unlock.}
    \label{fig:orphan.c}
\end{figure}